%\documentclass[journal]{IEEEtran}
\documentclass[11pt,onecolumn]{IEEEtran}
\ifCLASSINFOpdf
\else

\fi

\hyphenation{op-tical net-works semi-conduc-tor}


\begin{document}
\linespread{1.5}
\title{Summary of Two Papers}
\author{Yonglin~Zhuo\\
  Statistics and Data Mining\\
  Linkoping University\\
  Linkoping, Sweden\\
  yonzh457@student.liu.se}
\maketitle


\IEEEpeerreviewmaketitle

``Huge Data Sets and the Frontiers of Computational Feasibility'' by Edward J. Wegman\cite{hugedata}, gives some new perspectives of computational feasibility for algorithms. Firstly, the author transfers his attention from the data size to the impact of different magnitude orders. He simulates the computational feasibility of 4 CPU stations under different data sizes such as tiny $10^2$, small $10^4$, medium $10^6$, largr$10^8$ and huge$10^{10}$. As he concludes that which sizes of data can be handled by which computers. Then, it can be concluded that the interactive consumption is acceptable for the ``medium” data set and maybe ``large'' set saved in RAM. In addition, by analyzing the visualization accessing the author proposes a conclusion that the visualization of data sets is hardly feasible when the sizes of data are $10^6$and more. Finally, the author sketches out some practical approaches such as recursion, adaptation, and designed sampling to release the negative influence of data size.\\

``Adaptive Fraud Detection'' by Tom Fawcett and Foster Provost\cite{Fraud} proposes a new approach for detecting and evaluation the illicit usage of an account called profiling method. And the experiments on the cellular cloning fraud database shows this method gets a preferable results rather than hand-crafted techniques. As we can see there are a few existing approaches that can handle cloning problems, for instance, pre-call methods and post-call methods\cite{Fraud}. Roughly speaking, pre-call detecting is strictly authenticating once communicating and post-call technique will circularly check a account if the fraud happens based on some records like collisions and velocity. It is also argued in the article that the user profiling technique\cite{Fraud} is a kind of post-call method. It includes three procedures. The first step is called learning fraud rules\cite{Fraud} which generates and selects the fraud indicators. Then in the constructing profiling monitors stage, we can establish profiling monitors processing one account-day for each time. At last, in the combining the evidence process, the author declares that a linear threshold unit method is launched for merging the evidence in the following experiments. Finally, error cost is used as the metric of evaluation. Even though it is concluded from the experiment results that the DC-1 and DC-1 complex methods is much better than the other methods such as collisions+velocities, high usage, and STOA\cite{Fraud}, fraud detection is a cutting-edge domain that needs more study in the future work.
\ifCLASSOPTIONcaptionsoff
  \newpage
\fi


\renewcommand\refname{Reference}
\small
\bibliographystyle{IEEEtran}
\bibliography{Bib}


\end{document}